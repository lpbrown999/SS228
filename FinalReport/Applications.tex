\section{Applications to Super Smash Bros Melee}

\subsection{Discretized actions}
The controller for the Nintendo game cube can be thought of as a continuous action space for Super Smash Brothers Melee. There are two analog control sticks which can be placed at any value from -1 to 1 in both the x and y direction, two analog shoulder buttons which are functionally identical and range from 0 to 1, as well as four digital face buttons. 

To reduce the number of potential actions, repetitive combinations of buttons were discarded and the analog inputs were discretized. We represented five buttons ($A$,$B$,$X$,$L$, and $Z$) as binary variables, where a value of one maps to pressed and zero maps to not pressed, and the analog stick as a variable with three possible values for both the x direction, $N^{x}$, and the y direction, $N^{y}$. We also include a "short hop" button press, $X_{s}$, where X is pressed for a shorter amount of time, and a "none" button press, $\varnothing$, where none of the buttons are pressed. 

To create an action, we select a single button, an x direction for the analog stick, and a y direction for the analog stick. With seven possible button presses ($A$,$B$,$X$,$X_{s}$,$L$,$Z$, $\varnothing$) and three possible tilt angles in each stick direction, this results in an action space $\mathbb{A}$ 63 discrete actions.

\subsection{Basis Functions}
As our project is based on, perceptron Q-learning we designed a set of basis functions to span the state space in Super Smash Brothers Melee and allow for perceptron based approximation of state action values. Our beta function, $\beta$, contains the following elements.

\begin{itemize}
\item A set of normal distributions along the x and y axes
\item A set of normal distributions for the relative distance between the agent and the opponent
\item A flag for the direction the agent is facing and a flag for the direction the opponent is facing
\item A set of flags for the agent and a set of flags for the opponent to represent unique animations
\item A set of normal distributions for the agent and opponents damages
\item A set of flags for the number of jumps left
\end{itemize}

Additionally, the state space is discretized into three "super-states", being off the stage to the left, on the stage, and off the stage to the right. A zero padded vector $\beta_{p} = [\beta,~0_{|\beta|},~0_{|\beta|}]$ is then created, and the base $\beta$ function is inserted to the appropriate super-state index. The need and  justification for this technique is discussed in the novel approaches section.

\subsection{Reward Functions}

In order to apply perceptron Q-learning to to SSBM, a reward function was defined. As described in the introduction, the goal of the game is to ultimately knock your opponent off the stage, which becomes easier as their damage increases since the damage causes them to fly further. This indicates a careful balance between accumulating damage to facilitate a knock out move, and overly focusing on accumulating damage with no regard for the ultimate goal of getting a knock out. To accomplish this, we defined the reward function to favor dealing damage to the opponent when they are at lower damage, decaying for damage dealt when the opponent is already at higher percentages. Additionally, the agent receives a penalty for taking damage in the same fashion as dealing damage. To prevent the agent from jumping off the stage and prematurely dying, a reward was also given when the agent managed to move from being off of the side of the stage in state s, to on the stage in state $s'$. Rewards are also assigned for kills and penalty was imposed for dying.

Denoting the opponents damage as $d_o$, the agents damage as $d_a$, and the on stage parameter of the agent as a true / false flag $ON$, the reward function becomes:

$$R = (d'_o-d_o)exp(-.01*d_o) - (d'_a-d_a)exp(-.01*d_a) + r_{jump}\delta_{0,ON}\delta_{1,ON'}  + r_{kill} + r_{death}$$

The reward assignment was not straight forward due to the lag in state-evolution, action lockout, and delay between actions and resulting kills and deaths. The methods in which we handle these issues is described in the next section.
